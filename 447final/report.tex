
\documentclass[10pt]{article}

\usepackage[letterpaper, hmargin=0.3in, vmargin=0.3in]{geometry}
\usepackage{float}
\usepackage{url}
\usepackage{listings}
\usepackage{graphicx}
\usepackage{courier}
\usepackage{tikz}

\lstset{basicstyle=\footnotesize\ttfamily}
\renewcommand{\normalsize}{\fontsize{15pt}{\baselineskip}\selectfont}
\renewcommand{\baselinestretch}{1.3}
\usetikzlibrary{arrows,automata,shapes}
\title{CS447 Project}
\author{GROUP 6\\Yunfei Cui 20455776\\Junnan Chen 20595284}
\date{April 3, 2015}

\begin{document}
\maketitle
\section*{Part 1 c}
FOR CINDY!!!!

\section*{Part 2}
\subsection{Resolving Bugs in Apache Commons}
\subsubsection{CID:10022}
\textbf{Classification}
False Positive.
\\\textbf{Explain} 
The function here returns a KeySetView object and doesn't need to call super.keyset() to get the keyset of map.

\subsubsection{CID:10023}
\textbf{Classification}
False Positive.
\\\textbf{Explain} 
In this case, the function map.keyset() is the same function as super.keyset().

\subsubsection{CID:10024}
\textbf{Classification}
False Positive.
\\\textbf{Explain} 
Same as the previous one.

\subsubsection{CID:10025}
\textbf{Classification}
False Positive.
\\\textbf{Explain} 
The currentIterator has to be checked before because in different branches, the program contains other if-statements and should execute different code.

\subsubsection{CID:10026}
\textbf{Classification}
Bug
\\\textbf{Explain} 
The get function reads the list l, integer first and last without holding lock. Some of the usages didn't require a lock before calling this function. That might cause a conflict or fault. 
\\\textbf{Suggested Fix}
\begin{verbatim}
private List get(List l) {
             if (list != expected) {
                 throw new ConcurrentModificationException();
             }
             	synchronized(FastArrayList.this)
             	{
            		List tore = l.subList(first, last);
             		return l.subList(first, last);
             	}
         }
\end{verbatim}

\subsubsection{CID:10027}
\textbf{Classification}
False Positive.
\\\textbf{Explain} 
nextGreater function will not return null because we have checked that (deletedNode.getRight(index) != null) and it will go to the first else if (line 622, TreeBidiMap.java) and return leastNode.

\subsubsection{CID:10028}
\textbf{Classification}
Bug
\\\textbf{Explain}
To avoid synchronous modification, the parameter last should acquire a lock when used.
\\\textbf{Suggested Fix}
\begin{verbatim}
public void add(Object o) {
           checkMod();
           synchronized(FastArrayList.this){
    	        int i = nextIndex();
  	          get().add(i, o);
        		  last++;
        	    	  expected = list;
        	    	  iter = get().listIterator(i + 1);
        	    	  lastReturnedIndex = -1;
            }
        }
\end{verbatim}

\subsubsection{CID:10029}
\textbf{Classification}
Bug
\\\textbf{Explain}
If two threads execute the function synchronously, both of them will pass the first check(lastReturned == null) and try to acquire the lock. After the first thread succeeds removing the list, parameter lastRetured is set to null. In this case, the second thread should throw IllegalStateException, but it already passed that check.
\\\textbf{Suggested Fix}
\begin{verbatim}
public void remove() {
 	synchronized (FastHashMap.this) {
                if (lastReturned == null) {
                    throw new IllegalStateException();
                }
                if (fast) {
                        if (expected != map) {
                            throw new ConcurrentModificationException();
                        }
                        FastHashMap.this.remove(lastReturned.getKey());
                        lastReturned = null;
                        expected = map;
                } else {
                    iterator.remove();
                    lastReturned = null;
                }
      }
}
\end{verbatim}



\subsubsection{CID:10030}
\textbf{Classification}
False Positive.
\\\textbf{Explain} 
Normally, this is not a problem because the parameter lock is protected, only the member functions can use lock. And all the other member functions won't require to synchronized(map) outside the synchronized(lock). So there won't be a deadlock.

\subsubsection{CID:10031}
\textbf{Classification}
False Positive.
\\\textbf{Explain} 
the function rotateLeft didn't check if the current node has a rightsubtree. But the function is private and is called only when the current node has a rightsubtree that is 2 levels deeper than its leftsubtree because of the definition of AVL tree.


\subsubsection{CID:10032}
\textbf{Classification}
Bug
\\\textbf{Explain}
If two threads execute the function synchronously, both of them will pass the first check(bucket $<$ buckets.length) and try to acquire the lock. After the first thread succeeds removing the list, parameter bucket adds 1. If after the adding ,bucket equals to buckets.length, in this case, the second thread shouldn't pass the check, but it already passed that check.
\\\textbf{Suggested Fix}
\begin{verbatim}
public boolean hasNext() {
            if (current.size() > 0) return true;
            synchronized(bucket){
            while (bucket < buckets.length) {
                synchronized (locks[bucket]) {
                    Node n = buckets[bucket];
                    while (n != null) {
                        current.add(n);
                        n = n.next;
                    }
                    bucket++;
                    if (current.size() > 0) return true;
                }
            }
            }
            return false;
        }
\end{verbatim}

\subsubsection{CID:10033}
\textbf{Classification}
False Positive.
\\\textbf{Explain}
This is not a bug because the function is in "INVERSE MAP ENTRY", what we need here is to get the reversed entry. The value should be passed as a key and the key should be passed as a value.

\subsubsection{CID:10034}
\textbf{Classification}
Bug
\\\textbf{Explain}
Same as CID 10032.
\\\textbf{Suggested Fix}
\begin{verbatim}
public boolean hasNext() {
            if (current.size() > 0) return true;
            synchronized(m_bucket){
            while (bucket < m_buckets.length) {
                synchronized (m_locks[bucket]) {
                    Node n = m_buckets[bucket];
                    while (n != null) {
                        current.add(n);
                        n = n.next;
                    }
                    bucket++;
                    if (current.size() > 0) return true;
                }
            }
            }
            return false;
        }
\end{verbatim}

\subsubsection{CID:10035}
\textbf{Classification}
Bug
\\\textbf{Explain}
To avoid synchronous modification, the parameter last should acquire a lock when used.
\\\textbf{Suggested Fix}
\begin{verbatim}
public void remove() {
            checkMod();
            if (lastReturnedIndex < 0) {
                throw new IllegalStateException();
            }
            synchronized(FastArrayList.this){
            get().remove(lastReturnedIndex);
            last--;
            expected = list;
            iter = get().listIterator(lastReturnedIndex);
            lastReturnedIndex = -1;
            }
        }
\end{verbatim}

\subsubsection{CID:10036}
\textbf{Classification}
Bug
\\\textbf{Explain}
Same as CID 10029.
\\\textbf{Suggested Fix}
\begin{verbatim}
public void remove() {
 	synchronized (FastHashMap.this) {
                if (lastReturned == null) {
                    throw new IllegalStateException();
                }
                if (fast) {
                        if (expected != map) {
                            throw new ConcurrentModificationException();
                        }
                        FastHashMap.this.remove(lastReturned.getKey());
                        lastReturned = null;
                        expected = map;
                } else {
                    iterator.remove();
                    lastReturned = null;
                }
      }
}
\end{verbatim}

\subsubsection{CID:10037}
\textbf{Classification}
False Positive.
\\\textbf{Explain}
Similar with CID:10031


\subsubsection{CID:10038}
\textbf{Classification}
False Positive.
\\\textbf{Explain} 
Normally, this is not a problem because the parameter lock is protected, only the member functions can use lock. And all the other member functions won't require to synchronized(list) outside the synchronized(lock). So there won't be a deadlock.

\subsubsection{CID:10039}
\textbf{Classification}
False Positive.
\\\textbf{Explain} 
same as CID 10027.

\subsubsection{CID:10040}
\textbf{Classification}
False Positive.
\\\textbf{Explain} 
Normally, this is not a problem because the parameter lock is protected, only the member functions can use lock. And all the other member functions won't require to synchronized(map) outside the synchronized(lock). So there won't be a deadlock.

\subsubsection{CID:10041}
\textbf{Classification}
Intentional
\\\textbf{Explain} 
In this case, all the member functions are not locked in ReferenceMap. The developer intentionally wrote it this way because for most single threads, locks are not necessary, and for multi-threads users, the developer wrote comments to remind them add a lock when calling the corresponding functions.
\\\textbf{Bad Practice}
thread1 reads the value of original modCount as v1. Control switches to thread2. Thread2 reads value of original modCount as v1. Control switches to thread1. Thread1 changes the value of modCount to v1+1. Control switches to thread2. Thread2 changes the value of modCount to v1+1. This is a fault and the correct value should be v1+2.

\subsubsection{CID:10042}
\textbf{Classification}
Intentional
\textbf{Explain \& Bad Practice} 
The same reason as the previous one(CID:10041).


\section*{Analyze Own Code}
\textbf{Total Bug:}
1.\\
\textbf{Classification}
Bug.\\
\textbf{Explain}
Argument fixed will change the output format of cerr. It should be reset after use.\\
\textbf{Fix}
\begin{verbatim}
ios::fmtflags f(cerr.flags());
//code
cerr.flags(f);
\end{verbatim}
\end{document}





















