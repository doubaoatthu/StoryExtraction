
\documentclass[12pt]{article}

\usepackage[letterpaper, hmargin=0.95in, vmargin=0.95in]{geometry}
\usepackage{float}
\usepackage{url}
\usepackage{listings}
\usepackage{graphicx}
\usepackage{courier}
\usepackage{tikz}

\lstset{basicstyle=\footnotesize\ttfamily}
\renewcommand{\normalsize}{\fontsize{15pt}{\baselineskip}\selectfont}
\renewcommand{\baselinestretch}{1.3}
\usetikzlibrary{arrows,automata,shapes}
\title{Chinese Story Extraction by \\Information Distance Models in Large Tour Guide Document}
\author{Junnan CHEN (20595284), {\tt j486chen@uwaterloo.ca}}
\date{April 3, 2015}

\begin{document}
\maketitle

\section*{Introduction}
The Internet makes available a tremendous amount of text that has been generated for human consumption. In most cases, only a small part of the large text is useful and needed.  Unfortunately, this information is not easily manipulated or analyzed by computers. \emph{Chinese Story extraction} (CSE) is the process of filling fields in a databases by automatically extract Chinese story fragments of human-readable text. Example include extracting specific story of a person, a place and an object from a large document which contains redundant information.

\section Title Retrieve

\end{document}





















